\documentclass[dtu]{dtuarticle}
\usepackage{parskip} % use enters instead of indents

\usepackage{amsmath}
\usepackage{amssymb}
\usepackage{bm}
\usepackage{url}
\usepackage{hyperref}
\usepackage{subcaption}
\usepackage{siunitx}
\usepackage[most]{tcolorbox} % most is required for breakable
\usepackage{listings}
\usepackage{color}

\newcommand{\N}{\mathbb{N}}
\newcommand{\Z}{\mathbb{Z}}
\newcommand{\Zn}{\mathbb{Z}_n}
\newcommand{\Zp}{\mathbb{Z}_{\geq 0}}
\newcommand{\Zpp}{\mathbb{Z}_{>0}}
\newcommand{\Q}{\mathbb{Q}}
\newcommand{\R}{\mathbb{R}}
\newcommand{\C}{\mathbb{C}}
\newcommand{\F}{\mathbb{F}}
\newcommand{\E}{\mathbb{E}}
\newcommand{\eqDef}{\overset{\Delta}{=}}
\newcommand{\plusn}{+_n}
\newcommand{\timesn}{\cdot_n}
\newcommand{\id}{\mathrm{id}}
\newcommand{\gen}[1]{\langle #1 \rangle} % subgroup generated by #1
\newcommand{\red}[1]{\color{red}#1\color{black}}
\newcommand{\ord}{\operatorname{ord}}
\newcommand{\lcm}{\operatorname{lcm}}
\newcommand{\im}{\operatorname{im}}

\newcounter{qnumber}[section]
% \renewcommand{\theqnumber}{\thesubsection.\arabic{qnumber}}
\newenvironment{question}[1][]{
    \def\qpoints{#1}
    \refstepcounter{qnumber}
    \par\medskip\noindent    % Starts new para, adds space, removes indent
    \textbf{Question \theqnumber:}
}{
    % Check if the argument is empty
    \ifx\qpoints\empty
    \else
        \textbf{(\qpoints\ points)}
    \fi
    \par\medskip % Adds space after the question
}

\newcounter{choice}
\renewcommand\thechoice{\Alph{choice}}
\newcommand\choicelabel{\thechoice.}

\newenvironment{choices}%
  {\list{\choicelabel}%
     {\usecounter{choice}\def\makelabel##1{\hss\llap{##1}}%
       \settowidth{\leftmargin}{W.\hskip\labelsep\hskip 2.5em}%
       \def\choice{%
         \item
       } % choice
       \labelwidth\leftmargin\advance\labelwidth-\labelsep
       \topsep=0pt
       \partopsep=0pt
     }%
  }%
  {\endlist}

\newenvironment{oneparchoices}%
  {%
    \setcounter{choice}{0}%
    \def\choice{%
      \refstepcounter{choice}%
      \ifnum\value{choice}>1\relax
        \penalty -50\hskip 1em plus 1em\relax
      \fi
      \choicelabel
      \nobreak\enskip
    }% choice
    \def\rightchoice{
        \refstepcounter{choice}
        \ifnum\value{choice}>1\relax
            \penalty -50\hskip 1em plus 1em\relax
        \fi
        \textbf{\choicelabel}
        \nobreak\enskip
    }
    % If we're continuing the paragraph containing the question,
    % then leave a bit of space before the first choice:
    \ifvmode\else\enskip\fi
    \ignorespaces
  }%
  {}

\newtcolorbox{answer}{
    breakable,             % Allows splitting across pages
    colback=white,         % Background color
    colframe=gray,        % Border color
    % only top rule
    boxrule=0mm,
    toprule=0.2mm,
    width=\dimexpr\textwidth\relax, % Set width
    arc=0pt, outer arc=0pt,% Makes corners sharp (like tabular)
    left=0.0cm, right=0.0cm,   % Padding inside the box
    top=0.2cm, bottom=0.0cm,   % Padding inside the box
    parbox=false,          % Uses standard paragraph mode (better spacing)
    before={\textcolor{gray}{\sffamily\bfseries\footnotesize ANSWER}\vspace{0.1cm}}
}

\title{Exam Questions}
\subtitle{Introductory Economics}
\author{Vincent Van Schependom}
\course{42009 Introductory Economics}
\address{
	DTU Management \\
	Fall 2025
}
\date{Fall 2025}



\begin{document}

\maketitle

\begin{question}
    Which of the following activities have zero opportunity cost?
    \begin{choices}
        \choice To see a concert with a free ticket obtained from a lucky draw
        \choice To attend college fully on scholarships
        \choice To participate in the demonstration for equality in Nyhavn, Copenhagen
        \choice \textbf{None of the above. All of the events above have opportunity costs.}
    \end{choices}
\end{question}

\begin{question}
    Changes in the price of a good lead to:
    \begin{choices}
        \choice \textbf{Change in the quantity supplied of the good.}
        \choice Changes in supply.
        \choice Changes in demand.
        \choice No effects in quantity supplied or demanded.
    \end{choices}
\end{question}

\begin{answer}
    \begin{itemize}
        \item[A.] Changing \textit{only}
              the price leads to changes in the \textbf{quantity supplied}. This is graphically
              represented as a movement \textit{along} a given supply curve.
              Hence, \textbf{answer A} is correct.
              $$P_X \text{ changes } \Rightarrow Q_X^S \text{ changes as a function of } P_X$$

        \item[B/C.] `Supply' and demand' are the entire curves/relationships, not single points.
              A change in \textit{supply} refers to a \textbf{shift} of the
              \textit{entire} supply curve. This is caused by changing factors \textit{other} than the price of the
              good itself, such as input prices, technology, the number of firms, or taxes.
              Similary, a change in \textit{demand} refers to a \textbf{shift}
              of the \textit{entire} demand curve caused by factors like income, tastes, or prices of related goods.
              A change in the good's own price leads to a change in the \textit{quantity demanded}
              (movement along the curve), not a change in demand.

        \item[D.] Economic theory (Law of Supply and Law of Demand)
              states that price is the primary determinant of both quantity supplied and quantity demanded;
              therefore, a price change definitely has an effect.
    \end{itemize}
\end{answer}

\begin{question}
    To solve the traffic congestion problem in Copenhagen, the government is considering policies to reduce private cars on the road.
    Which of the following policy will \textbf{NOT} be helpful in achieving this goal?
    \begin{choices}
        \choice To provide subsidy to the people who ride a bike.
        \choice To implement a higher tax on car purchase.
        \choice To implement a higher tax on gasoline.
        \choice \textbf{To charge a higher price on public transport.}
    \end{choices}
\end{question}

\begin{question}
    The quantity demanded of a good decreases by 20\% when its price increase by 2\%.
    Which of the followings best fit this good?
    \begin{choices}
        \choice This good has no close substitutes.
        \choice This is a necessity good.
        \choice \textbf{This is more likely to happen in a short-run.}
        \choice This is an inferior good.
    \end{choices}
\end{question}

\begin{answer}
    The definition of an inferior good is a good for which $I \uparrow \implies Q_X^d \downarrow$\\
    On the other hand, a normal good satisfies $I \uparrow \implies Q_X^d \uparrow$\\
    Neither of these are applicable here.

    A necessity good is a good is an elastic good: $|E_{Q_X^d,P_X}| < 1$\\
    A luxury good is a good that is inelastic: $|E_{Q_X^d,P_X}| > 1$
    The price elasticity of demand in this case is
    $$|E_{Q_X^d,P_X}| = \frac{\Delta Q_X^d / Q_X^d}{\Delta P_X / P_X} = \frac{-0.2}{0.02} = -10$$
    Because $|E_{Q_X^d,P_X}| > 1$, this good is a luxury good.

    This is more likely to happen in a short-run because ...
\end{answer}

\begin{question}
    Suppose we observe a decrease of the equilibrium price of potato and an increase of the equilibrium quantity.
    Which of the following best fit the observed data?
    \begin{choices}
        \choice An increase in demand with supply unchanged
        \choice A decrease in supply with demand unchanged
        \choice \textbf{An increase in supply with demand unchanged}
        \choice An increase in demand coupled with a decrease in supply
    \end{choices}
\end{question}

\begin{answer}
    Equilibrium quantity moves in the same direction as the curve shift for both supply and demand, but price moves in opposite directions:
    \begin{itemize}
        \item Demand $\uparrow$: $P \uparrow, Q \uparrow$
        \item Supply $\uparrow$: $P \downarrow, Q \uparrow$
    \end{itemize}
    Since $P$ fell and $Q$ rose, this must be a positive supply shift.
\end{answer}

\begin{question}
    The cross price elasticity between good x and good y is found to be positive. We conclude that good x and good y are:
    \begin{choices}
        \choice normal goods
        \choice inferior goods
        \choice \textbf{substitutes}
        \choice complements
    \end{choices}
\end{question}

\begin{answer}
    Cross price elasticity $\epsilon_{xy} > 0$ means that if the price of $Y$ goes up,
    the demand for $X$ goes up. This implies consumers are switching from $Y$ to $X$,
    making them substitutes.
\end{answer}

\begin{question}
    The CEO of a large restaurant chain said, "For each 1 percent price increase, we lose 5 percent of our diners." We can conclude that:
    \begin{choices}
        \choice demand is price inelastic.
        \choice a price increase will decrease total revenue.
        \choice the price elasticity is -0.5.
        \choice \textbf{a price increase will decrease total revenue.}
    \end{choices}
\end{question}

\begin{answer}
    The elasticity is $\epsilon = \frac{-5\%}{+1\%} = -5$. Since $|\epsilon| > 1$, demand is elastic.
    When demand is elastic, the percentage drop in quantity outweighs the percentage rise in price, causing Total Revenue ($P \times Q$) to fall.
    (Note: While the elasticity is technically -5, not -0.5, the most robust inference about revenue is the correct choice here).
\end{answer}

\begin{question}
    Suppose the (inverse) demand for a product is $P=40-Q$ and (inverse) supply of the product is $P=4+2Q$. The equilibrium quantity, price, and consumer surplus (CS) would be:
    \begin{choices}
        \choice $Q=12, P=28, CS=72$
        \choice $Q=8, P=14, CS=36$
        \choice \textbf{$Q=12, P=28, CS=72$}
        \choice $Q=8, P=14, CS=72$
    \end{choices}
\end{question}

\begin{answer}
    Set Demand = Supply:
    $$40 - Q = 4 + 2Q \implies 36 = 3Q \implies Q = 12$$
    Substitute $Q$ back to find $P$:
    $$P = 40 - 12 = 28$$
    Consumer Surplus is the area of the triangle below the demand intercept ($P=40$) and above the market price ($P=28$):
    $$CS = 0.5 \times \text{Base} \times \text{Height} = 0.5 \times 12 \times (40-28) = 0.5 \times 12 \times 12 = 72$$
\end{answer}

\begin{question}
    A price ceiling is
    \begin{choices}
        \choice the minimum legal price that can be charged in a market.
        \choice \textbf{the maximum legal price that can be charged in a market.}
        \choice higher than the initial equilibrium price.
        \choice equal to the initial equilibrium price.
    \end{choices}
\end{question}

\begin{question}
    Suppose the production function is given by $Q=\min[3K,4L]$. What is name of this production function form and what is the average product of labor $(AP_{L})$ when 15 units of capital and 10 units of labor are employed?
    \begin{choices}
        \choice Cobb-Douglas, $AP_{L}=4$
        \choice \textbf{Leontief, $AP_{L}=4$}
        \choice Cobb-Douglas, $AP_{L}=3$
        \choice Leontief, $AP_{L}=3$
    \end{choices}
\end{question}

\begin{answer}
    The form $Q=\min[aK, bL]$ is a **Leontief** (fixed proportion) function.
    Substitute $K=15, L=10$:
    $$Q = \min[3(15), 4(10)] = \min[45, 40] = 40$$
    Average Product of Labor:
    $$AP_L = \frac{Q}{L} = \frac{40}{10} = 4$$
\end{answer}

\begin{question}
    Suppose the production function is $Q=3K+4L$. The marginal rate of technical substitution is:
    \begin{choices}
        \choice \textbf{4/3}
        \choice 2/3
        \choice 8/3
        \choice 5/6
    \end{choices}
\end{question}

\begin{answer}
    For a linear production function $Q = aK + bL$, the marginal products are constant: $MP_K = 3$ and $MP_L = 4$.
    $$MRTS_{LK} = \frac{MP_L}{MP_K} = \frac{4}{3}$$
\end{answer}

\begin{question}
    A firm uses labor (L) and capital (K) as inputs to produce. If the price of inputs are $w=60$ DKK, $r = 200$ DKK, and marginal products are $MP_{L}=30, MP_{K}=100$, the firm:
    \begin{choices}
        \choice \textbf{is cost minimizing.}
        \choice should use less L and more K to cost minimize.
        \choice should use less K and more L to cost minimize.
        \choice is profit maximizing but not cost minimizing.
    \end{choices}
\end{question}

\begin{answer}
    Cost minimization requires the "bang for the buck" to be equal across inputs:
    $$\frac{MP_L}{w} = \frac{30}{60} = 0.5$$
    $$\frac{MP_K}{r} = \frac{100}{200} = 0.5$$
    Since the ratios are equal, the firm is optimizing.
\end{answer}

\begin{question}
    Regarding isoquants and isocosts, which of the followings is NOT correct?
    \begin{choices}
        \choice An isoquant defines the combinations of inputs that yield the producer the same level of output.
        \choice An isocost line defines the combinations of inputs that yield the producer the same cost.
        \choice \textbf{An isoquant should never intersect with an isocost line.}
        \choice The producer is cost minimizing at the point of tangency between an isoquant and an isocost line.
    \end{choices}
\end{question}

\begin{question}
    Regarding the average and marginal costs, which of the following is NOT correct?
    \begin{choices}
        \choice Average total cost increases when marginal cost curve is above the average total cost curve.
        \choice The marginal cost curve intersects the average total cost curve at the minimum point of average total cost curve.
        \choice The marginal cost curve intersects the average variable cost curve at the minimum point of average variable cost curve.
        \choice \textbf{Marginal cost decreases when average fixed cost decreases.}
    \end{choices}
\end{question}

\begin{answer}
    Average Fixed Cost (AFC) \textit{always} decreases as output increases.
    Marginal Cost (MC), however, is typically U-shaped (it eventually rises due to diminishing returns).
    There is no rule stating MC must fall just because AFC is falling.
    The other three statements are fundamental laws of cost curves.
\end{answer}

\begin{question}
    Economies of scale exist when
    \begin{choices}
        \choice \textbf{average total costs decline as output increases.}
        \choice average total costs increase as output increases.
        \choice average total costs remains constant as output increases.
        \choice average fixed costs decline as output increases.
    \end{choices}
\end{question}

\begin{question}
    Generally, an increase in the number of vegetarians will cause the demand curve for meat to
    \begin{choices}
        \choice shift rightward.
        \choice \textbf{shift leftward.}
        \choice become flatter.
        \choice become steeper.
    \end{choices}
\end{question}

\begin{question}
    Which of the following would cause the current supply curve of iPhone to shift rightward?
    \begin{choices}
        \choice an economic boom, which increases the amount that people are willing to spend on personal electronics
        \choice a decrease in the price of songs on Apple music
        \choice \textbf{the producer's expectation that the future price of iPhone will decrease}
        \choice an increase in the wages of labor in iPhone manufacturers
    \end{choices}
\end{question}

\begin{answer}
    \begin{itemize}
        \item A and B affect \textbf{demand}, not supply.
        \item D (higher wages) increases costs, shifting supply \textbf{left}.
        \item C suggests sellers want to sell now before prices drop, increasing current supply (shifting \textbf{right}).
    \end{itemize}
\end{answer}

\begin{question}
    Regarding accounting and economic profits/costs, which of the following is NOT correct?
    \begin{choices}
        \choice Accounting profits generally overstate economic profits.
        \choice Accounting profits do not take opportunity cost into account.
        \choice Economic costs include not only the accounting costs but also the opportunity costs of the resources used in production.
        \choice \textbf{Managers should only care about accounting profits.}
    \end{choices}
\end{question}

\begin{question}
    A firm in a competitive market sells its product at a price of 60 DKK and its cost function is $C(Q)=20+5Q^{2}.$ The maximum profits for the firm would be:
    \begin{choices}
        \choice \textbf{160 DKK}
        \choice 100 DKK
        \choice 360 DKK
        \choice 200 DKK
    \end{choices}
\end{question}

\begin{answer}
    Profit maximization in perfect competition occurs where $P = MC$.
    $$MC = \frac{dC}{dQ} = 10Q$$
    $$60 = 10Q \implies Q = 6$$
    Now calculate Profit ($TR - TC$):
    $$TR = 60 \times 6 = 360$$
    $$TC = 20 + 5(6)^2 = 20 + 5(36) = 200$$
    $$\text{Profit} = 360 - 200 = 160$$
\end{answer}

\begin{question}
    Suppose you are a supervisor of PhD student, which of the following is NOT a solution to the principle-agent problem of supervisor-student?
    \begin{choices}
        \choice To give bonus for publication in journals/conferences
        \choice To give bonus for project reports
        \choice Spot checks at the office of PhD student
        \choice \textbf{Fixed salary regardless of performance}
    \end{choices}
\end{question}

\begin{question}
    Regarding fixed costs and sunk costs, which of the following is NOT correct?
    \begin{choices}
        \choice Sunk costs are those costs that are forever lost after they have been paid.
        \choice Fixed costs do not vary with output.
        \choice A lost ticket before a movie starts is an example of sunk cost.
        \choice \textbf{Sunk costs could be part of the marginal costs.}
    \end{choices}
\end{question}

\begin{answer}
    Marginal cost looks forward (the cost of the \textit{next} unit). Sunk costs are in the past.
    Therefore, sunk costs can never be part of marginal costs.
\end{answer}

\begin{question}
    The sources of monopoly power for a monopoly could be:
    \begin{choices}
        \choice economies of scale.
        \choice economies of scope.
        \choice patents.
        \choice \textbf{all of the above.}
    \end{choices}
\end{question}

\begin{question}
    The long-run equilibrium of a perfectly competitive market is characterized by:
    \begin{choices}
        \choice $P >$ minimum of ATC
        \choice $P <$ AVC
        \choice $P > MR$
        \choice \textbf{P = minimum of ATC (average total cost)}
    \end{choices}
\end{question}

\begin{question}
    A perfectly competitive firm will shut down (stop producing) when:
    \begin{choices}
        \choice market price is lower than the average total cost (ATC).
        \choice total revenue is less than the total cost.
        \choice \textbf{market price is lower than the average variable cost (AVC).}
        \choice fixed cost is too high.
    \end{choices}
\end{question}

\begin{question}
    Which of the following statements about a monopoly is NOT correct?
    \begin{choices}
        \choice A monopoly does not have a supply curve.
        \choice The demand curve of a monopoly is the market demand curve.
        \choice \textbf{A monopoly has no market power.}
        \choice A monopoly produces at $MR=MC$.
    \end{choices}
\end{question}

\begin{question}
    A monopoly faces a demand curve described by $P=90-3Q$ and has a total cost of $C(Q)=5+10Q+Q^{2}.$ The profit-maximizing price for the monopoly is:
    \begin{choices}
        \choice \textbf{60}
        \choice 10
        \choice 30
        \choice 20
    \end{choices}
\end{question}

\begin{answer}
    \begin{enumerate}
        \item Derive MR: $P=90-3Q \implies MR = 90 - 6Q$.
        \item Derive MC: $C(Q)=5+10Q+Q^2 \implies MC = 10 + 2Q$.
        \item Set $MR=MC$:
              $$90 - 6Q = 10 + 2Q \implies 80 = 8Q \implies Q = 10$$
        \item Find Price:
              $$P = 90 - 3(10) = 60$$
    \end{enumerate}
\end{answer}

\begin{question}
    Which of the following is NOT an example of negative externalities?
    \begin{choices}
        \choice Air pollution from a factory
        \choice The neighbor's barking dog
        \choice Health risk to others from second-hand smoke
        \choice \textbf{Being vaccinated against Covid-19 protects not only you, but also the people around you.}
    \end{choices}
\end{question}

\begin{question}
    Consider a monopoly where the inverse demand for its product is given by $P=50-2Q$. Total costs for this monopolist is $C(Q)=100+2Q+Q^{2}$. At the profit-maximizing combination of output and price, deadweight loss is:
    \begin{choices}
        \choice \textbf{32}
        \choice 64
        \choice 128
        \choice cannot be determined with the given information.
    \end{choices}
\end{question}

\begin{answer}
    \textbf{Monopoly Outcome ($Q_m$):}
    $MR = 50 - 4Q$ and $MC = 2 + 2Q$.
    $$50 - 4Q = 2 + 2Q \implies 48 = 6Q \implies Q_m = 8$$
    $$P_m = 50 - 2(8) = 34$$
    \textbf{Social Optimum ($Q_{soc}$):}
    Set $P = MC$:
    $$50 - 2Q = 2 + 2Q \implies 48 = 4Q \implies Q_{soc} = 12$$
    \textbf{Deadweight Loss (DWL):}
    Area of triangle between $Q_m$ and $Q_{soc}$.
    Height of triangle at $Q_m=8$ is the difference between Demand and MC:
    $$P(8) - MC(8) = 34 - (2+2(8)) = 34 - 18 = 16$$
    $$DWL = 0.5 \times \text{Base} \times \text{Height} = 0.5 \times (12-8) \times 16 = 32$$
\end{answer}

\begin{question}
    Which of the following is NOT a transaction cost associated with using inputs?
    \begin{choices}
        \choice Time spent negotiating labor contracts with union workers
        \choice Opportunity costs of negotiating the price of renting machines.
        \choice \textbf{Wages paid to labor}
        \choice Costs of searching for new supplier of machines
    \end{choices}
\end{question}

\begin{answer}
    Transaction costs are the costs of making an economic exchange (search, bargaining, enforcement).
    The wage itself is the **price** of the input, not a transaction cost.
\end{answer}

\begin{question}
    In a free market in which an equilibrium price and quantity prevails:
    \begin{choices}
        \choice consumer surplus is less than producer surplus.
        \choice consumer surplus is greater than producer surplus.
        \choice consumer surplus is the same as producer surplus.
        \choice \textbf{the sum of consumer surplus and producer surplus are maximized.}
    \end{choices}
\end{question}

\begin{question}
    You are a division manager at Toyota. If your marketing department estimates that the semiannual demand for the Highlander is $Q=150,000-1.5P$, what price should you charge in order to maximize revenues from sales of the Highlander?
    \begin{choices}
        \choice \textbf{50,000}
        \choice 30,000
        \choice 100,000
        \choice 60,000
    \end{choices}
\end{question}

\begin{answer}
    Revenue is maximized at the midpoint of a linear demand curve (where elasticity = -1).
    Inverse demand: $1.5P = 150,000 - Q \implies P = 100,000 - \frac{2}{3}Q$.
    The intercept is 100,000. The midpoint price is half the intercept:
    $$P = \frac{100,000}{2} = 50,000$$
    Alternatively, solve $MR=0$.
\end{answer}

\begin{question}
    In a competitive market, the market demand is $Q^{d}=70-3P$ and the market supply is $Q^{S}=6P$. A price ceiling of 4 will result in a
    \begin{choices}
        \choice shortage of 24 units.
        \choice \textbf{shortage of 34 units.}
        \choice surplus of 24 units.
        \choice surplus of 34 units.
    \end{choices}
\end{question}

\begin{answer}
    First, check equilibrium price: $70-3P = 6P \implies 70 = 9P \implies P \approx 7.77$.
    Since the ceiling ($P=4$) is below equilibrium, it binds.
    $$Q_d = 70 - 3(4) = 58$$
    $$Q_s = 6(4) = 24$$
    $$\text{Shortage} = Q_d - Q_s = 58 - 24 = 34$$
\end{answer}

\begin{question}
    Andy, a college student, loves eating burger. As a college student with no income, he is used to eating at McDonald. After graduation, Andy gets a job. As such, his income is now 200,000 DKK a year. He ends up eating burger at Sporvejen. In economic terms, the burger at McDonald is a(n) \_\_\_, while the burger at Sporvejen is a(n) \_\_\_.
    \begin{choices}
        \choice normal good; normal good.
        \choice inferior good; inferior good.
        \choice normal good; inferior good.
        \choice \textbf{inferior good; normal good.}
    \end{choices}
\end{question}

\begin{question}
    Consider a monopoly where the inverse demand for its product is given by $P=100-3Q$. Base on this information, the marginal revenue function is:
    \begin{choices}
        \choice $MR(Q)=200-1.5Q$
        \choice \textbf{$MR(Q)=100-6Q$}
        \choice $MR(Q)=100-1.5Q$
        \choice $MR(Q)=200-6Q$
    \end{choices}
\end{question}

\begin{answer}
    For a linear demand curve $P = a - bQ$, the Marginal Revenue curve has the same intercept ($a$) but twice the slope ($2b$).
    $$P = 100 - 3Q \implies MR = 100 - 2(3)Q = 100 - 6Q$$
\end{answer}

\begin{question}
    The recipe that defines the maximum amount of output that can be produced with K units of capital and L units of labor is the
    \begin{choices}
        \choice \textbf{production function.}
        \choice cost function.
        \choice marginal product.
        \choice average product.
    \end{choices}
\end{question}

\begin{question}
    Suppose the demand for a product is $\ln Q^{d}=20-2 \ln P$, then this product is
    \begin{choices}
        \choice \textbf{elastic}
        \choice inelastic
        \choice unitary elastic
        \choice cannot be determined.
    \end{choices}
\end{question}

\begin{answer}
    This is a log-linear demand function of the form $Q = A P^\epsilon$. The coefficient of $\ln P$ is the price elasticity of demand.
    Here, $\epsilon = -2$. Since $|\epsilon| [cite_start]= 2 > 1$, the demand is **elastic** [cite: 393-397].
\end{answer}

\begin{question}
    When a demand curve is linear,
    \begin{choices}
        \choice the elasticity is the same as the slope of the demand curve.
        \choice \textbf{demand is elastic at high prices.}
        \choice demand is unitary elastic at low prices.
        \choice the elasticity is constant at all prices.
    \end{choices}
\end{question}

\begin{question}
    If a firm's production function is Leontief and the wage rates goes up, the
    \begin{choices}
        \choice \textbf{cost minimizing combination of capital and labor does not change.}
        \choice firm would use more labor to minimize the cost for a given output
        \choice firm would use more capital to minimize the cost for a given output
        \choice firm would use less labor to minimize the cost for a given output
    \end{choices}
\end{question}

\begin{answer}
    A Leontief production function has L-shaped isoquants (perfect complements).
    There is zero substitutability between inputs.
    Therefore, even if relative input prices change, the firm cannot substitute capital for labor; the optimal ratio remains fixed.
\end{answer}

\end{document}
